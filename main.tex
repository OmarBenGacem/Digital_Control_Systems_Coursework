\documentclass{article}
\usepackage{graphicx} % Required for inserting images

\title{Digital Control Systems Coursework: Omar Ben-Gacem}
\author{Omar Ben-Gacem}
\date{January 2025}



%%%%%%%%%%%%     Formatting     %%%%%%%%%%%%

\usepackage[sorting=none]{biblatex} %Imports biblatex package

\addbibresource{bibliography.bib} %Import the bibliography file
\usepackage[margin=1in,headsep=.5in]{geometry}
\usepackage{amsmath} % for bmatrix and pmatrix
\usepackage{amssymb}
\usepackage{wrapfig}
\usepackage{caption}
\usepackage{fancyhdr}
\usepackage{subcaption}
\usepackage{gensymb}
\usepackage{float}
\usepackage{ragged2e}
\usepackage{booktabs} % For better table formatting
\usepackage[export]{adjustbox}
\geometry{
 a4paper,
 total={170mm,257mm},
 left=19mm,
 right=19mm,
 top=23mm,
 bottom=23mm
 }
\renewcommand{\sectionmark}[1]{%
    \markright{\MakeUppercase{#1}}%
}



\title{Digital Control Systems Coursework Submission}
\author{Omar Ben-Gacem}
\pagestyle{fancy}
\fancyhead[RO]{{\rightmark}}
\date{November 2024}
\lhead{Omar Ben-Gacem\\01883771\\Digital Control Systems Coursework}






%%%%%%%%%%%%     Formatting     %%%%%%%%%%%%


\begin{document}


\section*{Executive Summary}
\begin{equation}
R = \left(\begin{array}{cccc} 0 & \frac{1}{M} & -\frac{F}{M^2} & \frac{F^2}{M^3}\\ \frac{1}{M} & -\frac{F}{M^2} & \frac{F^2}{M^3} & -\frac{F^3}{M^4}\\ 0 & -\frac{1}{L\,M} & \frac{F}{L\,M^2} & -\frac{g}{L^2\,M}-\frac{F^2}{L\,M^3}\\ -\frac{1}{L\,M} & \frac{F}{L\,M^2} & -\frac{g}{L^2\,M}-\frac{F^2}{L\,M^3} & \frac{\frac{F^3}{L\,M^3}+\frac{F\,g}{L^2\,M}}{M} \end{array}\right)
\end{equation}



\section{System Dynamics (Part A)}
The system being modeled is an inverted pendulum attached to a sliding cart. The target is to control the liner displacement of the card in order to hold the pendulum in an inverted position. Equation \ref{eqofm} gives the Newtonian equations of motion for the cart-pendulum system.

\[
\begin{cases}
  M \ddot{s}(t) + F \dot{s}(t) - \mu(t) & = 0 \\
    \ddot{\phi}(t) - \frac{g}{L} \ddot{s}(t) \sin({\phi}(t)) - \frac{\mu(t)}{L} \cos(\phi(t)) & = 0
\end{cases}
\]

\subsection{A1) Standard Form}
To write the cart-pendulum system in standard form, the state variable $\underline{x}$ and the input $\underline{u}$ must be identified. The applied force $\mu(t)$ acts as the only input to this system. By inspection of the equations of motion, it is trivial to show that displacement $s$ and angle $\phi$ both have first- and second-order differential terms, thus making state variable $\underline{x} = [s, \dot s, \phi, \dot\phi]^T$. Equation \ref{eqofm} can be rearranged and reduced to give Equation \ref{fofx} which acts as the state update function.

\begin{equation}\label{fofx}
    \dot{x}=\begin{bmatrix}\dot{s} \\ \ddot{s} \\ \dot{\phi} \\ \ddot{\phi}\end{bmatrix} = f(\underline{x},\underline{u}) = \n\left(\begin{array}{cccc} 0 & 1 & 0 & 0\\ 0 & -\frac{F}{M} & 0 & 0\\ 0 & 0 & 0 & 1\\ 0 & \frac{F}{LM} & \frac{g}{L} & 0 \end{array}\right)
\n \begin{bmatrix} s \\ \dot{s} \\ \phi \\ \dot{\phi}\end{bmatrix} +     \begin{bmatrix}
        0 \\
        \frac{1}{M} \\
        0 \\
        -\frac{1}{L M}
    \end{bmatrix}\,\mu(t)
\end{equation}

\subsection{A2) Free Response Equilibrium}
Setting $u=\mu(t)=0$ gives the free response of the system, where the equilibrium of this system are values of $\underline{\dot x}=\underline{0}$. The values of $\underline{x}$ that satisfy this criteria are shown below in Equation \ref{ueqzero}.

\begin{equation}\label{ueqzero}
\begin{bmatrix}
    \dot s \\ \ddot s \\ \dot \phi \\ \ddot \phi
\end{bmatrix} =  \begin{bmatrix}
    0 \\ 0 \\ 0 \\ 0
\end{bmatrix} = \begin{bmatrix}
    \dot{s} \\
    \frac{u}{M}-\frac{F\dot{s}}{M}  \\
    \dot{\phi} \\
     \frac{g\sin\left(\phi \right)}{L}-\frac{u\cos\left(\phi \right)}{LM}+\frac{F\dot{s}\cos\left(\phi \right)}{LM} \\
\end{bmatrix} \rightarrow \mu=0 \rightarrow \underline{x} = \begin{bmatrix}
    s \\ \dot s \\ \phi \\ \dot \phi
\end{bmatrix} = \begin{bmatrix}
    \Re \\
    0 \\
    \pi n, n \in \mathbb{Z} \\
    0 \\
\end{bmatrix}
\end{equation}



Equation \ref{fofx} shows that the displacement of the cart, $s$, has no impact on the systems ability to reach equilibrium. Thus the possible values of $s$ is determined to be the set of real numbers. In practice, it is bounded to be a reasonable amount of translational motion. The values of $\phi$ is set to be any period of $\pi$, meaning the pendulum is either vertically up or down. 

\subsection{A3) Linearization About an Equilibrium}
Given a point to linearize about, the state space model of the non-linear system $f(\underline{x},\underline{u})$ can be derived using the Jacobian operator to give the Jacobian matrix. The linearization should yeild an equation of the form $\dot x = Ax+Bu$. Equation \ref{jacobian} shows the taking of the Jacobian vector at the point $\underline{x}=\underline{0}$, to get the matrix $A$.

\begin{equation} \label{jacobian}
    A=  \left. J\{f(\underline{x},\underline{u})\} \right|_{\underline{x}=\underline{0}} =
    \left. \begin{bmatrix}
    \frac{\partial f_1(x,u)}{\partial x_1} & \frac{\partial f_1(x,u)}{\partial x_2} & \frac{\partial f_1(x,u)}{\partial x_3} & \frac{\partial f_1(x,u)}{\partial x_4} \\
    \frac{\partial f_2(x,u)}{\partial x_1} & \frac{\partial f_2(x,u)}{\partial x_2} & \frac{\partial f_2(x,u)}{\partial x_3} & \frac{\partial f_2(x,u)}{\partial x_4} \\
    \frac{\partial f_3(x,u)}{\partial x_1} & \frac{\partial f_3(x,u)}{\partial x_2} & \frac{\partial f_3(x,u)}{\partial x_3} & \frac{\partial f_3(x,u)}{\partial x_4} \\
    \frac{\partial f_4(x,u)}{\partial x_1} & \frac{\partial f_4(x,u)}{\partial x_2} & \frac{\partial f_4(x,u)}{\partial x_3} & \frac{\partial f_4(x,u)}{\partial x_4}
    \end{bmatrix} \right|_{\underline{x}=\underline{0}}
    = \n\left(\begin{array}{cccc} 0 & 1 & 0 & 0\\ 0 & -\frac{F}{M} & 0 & 0\\ 0 & 0 & 0 & 1\\ 0 & \frac{F}{LM} & \frac{g}{L} & 0 \end{array}\right)
\n
\end{equation}


To find matrix $B$, the Jacobian is taken again, however, this time with respect to input $u$. Equation \ref{jB} shows the derivation of the input Jacobian for equilibrium $\underline{x}=\underline{0}$.

\begin{equation} \label{jB}
    B=  \left. \frac{\partial f(\underline{x},\underline{u})}{\partial \underline{u}} \right|_{\underline{x}=\underline{0}} =
    \left. \begin{bmatrix}
    \frac{\partial f_1(x,u)}{\partial u} \\
    \frac{\partial f_2(x,u)}{\partial u} \\
    \frac{\partial f_3(x,u)}{\partial u} \\
    \frac{\partial f_4(x,u)}{\partial u}
    \end{bmatrix} \right|_{\underline{x}=\underline{0}}
    =     \begin{bmatrix}
        0 \\
        \frac{1}{M} \\
        0 \\
        -\frac{1}{L M}
    \end{bmatrix}
\end{equation}



This can be put into full state space representation as shown in Equation \ref{linized}. This gives the full state-space representation of the cart-pendulum system, and can be used to describe the local behavior of the system. about the point the equilibrium was taken.
\begin{equation}\label{linized}
    \underline{\dot{x}} = \begin{bmatrix} \dot{s} \\ \ddot{s} \\ \dot{\phi} \\ \ddot{\phi} \end{bmatrix} =\n\left(\begin{array}{cccc} 0 & 1 & 0 & 0\\ 0 & -\frac{F}{M} & 0 & 0\\ 0 & 0 & 0 & 1\\ 0 & \frac{F}{LM} & \frac{g}{L} & 0 \end{array}\right)
\n\begin{bmatrix} s \\ \dot{s} \\ \phi \\ \dot{\phi} \end{bmatrix}+     \begin{bmatrix}
        0 \\
        \frac{1}{M} \\
        0 \\
        -\frac{1}{L M}
    \end{bmatrix} [\mu]
\end{equation}

\subsection{A4) Linear Dynamics}
The  

\subsection{A5) Reachability}

\subsection{A6) Observability}

\end{document}
